\documentclass[]{main}
\fullname{Mohammad Hossein Nemati}
\jobtitle{DevOps Ingenieur}

\begin{document}
\resumeheader
{\phone{+98 937-758-8105}}
{\nationality{Iran, Teheran}}
{\email{ckoliber@gmail.com}}
{\website{ckoliber.com}}
{\linkedin{ckoliber}}
{\github{ckoliber}}

Erfahrener DevOps Ingenieur mit einem Master-Abschluss in Informatik von der Universität Teheran. Spezialisiert auf Hochleistungsrechnen, Infrastruktur als Code und CI/CD-Automatisierung. Bei Visiwise.co entwarf ich eine Cloud-Infrastruktur, die die Skalierbarkeit verbesserte und die Kosten um 20\% senkte. Bei Smartech verwaltete ich die Infrastruktur für 500 VMs und 50TB Daten und implementierte CI/CD-Pipelines, die die Build-Zeiten um 30\% reduzierten.

\begin{section}{Berufserfahrung}
 \begin{subsection}{Visiwise.co}{Senior DevOps Ingenieur}{Mai 2022 -- Gegenwart}{Teheran, Iran}
     \item Architekturierte Cloud-basierte Infrastruktur und optimierte bestehende Bare-Metal-Systeme
     \item Führte Datenbank-Migrationen ohne Ausfallzeiten durch und automatisierte Backups mit pgbackrest
     \item Implementierte Grafana-Monitoring-Stack mit benutzerdefinierten Dashboards und Alerts
     \item Orchestrierte Multi-Cloud-Ressourcen mit Terraform zur Kostenoptimierung und GitOps
     \item Startete selbst gehostete Dienste (N8n, Strapi, WordPress, Posthog, Sentry)
     \item Passte Quellcode an 12-Faktor-Standards an und entwickelte MR-Review-Apps
     \item Etablierte CI/CD-Pipelines mit GitlabCI und Docker Swarm
 \end{subsection}

 \begin{subsection}{Smartech}{Senior DevOps Ingenieur}{Feb 2024 -- Juni 2024}{Teheran, Iran}
     \item Verwaltete groß angelegte Infrastruktur mit über 500 VMs und Bare-Metal-Servern (50TB Daten)
     \item Wartete diverse Dienste (PostgreSQL, Elasticsearch, ClickHouse, ScyllaDB, Kafka, Redis, Minio)
     \item Standardisierte Projekte und entwarf CI/CD-Pipelines für verschiedene Technologien
     \item Erstellte Kubernetes-Manifeste und setzte ArgoCD für GitOps-basierte Prozesse ein
     \item Entwickelte Ansible-Rollen und -Playbooks für effiziente Service-Bereitstellung
 \end{subsection}

 \begin{subsection}{Universität Teheran}{DevOps Ingenieur}{Jul 2022 -- Jun 2023}{Teheran, Iran}
     \item Entwarf HPC-Cluster mit Slurm, Nvidia Enroot und Nvidia Pyxis
     \item Etablierte Slurm-Abrechnung, Quotenverwaltung und Überwachungssysteme
     \item Orchestrierte Cluster-Infrastruktur mit Terraform (IaC)
 \end{subsection}

 \begin{subsection}{Pishgam Vira}{DevOps Ingenieur}{März 2022 -- Juni 2022}{Teheran, Iran}
     \item Integrierte CI/CD-Pipelines mit GitHub Actions und Docker
     \item Einrichtung von Docker Swarm Bereitstellungsservern und automatisierte Releases mit semantic-release
 \end{subsection}

 \begin{subsection}{Mobtaker Darya}{Full-Stack Entwickler \& DevOps Ingenieur}{Nov 2019 -- Jun 2022}{Teheran, Iran}
     \item Leitete die Entwicklung des CCS-Projekts, einschließlich Crystal Reports und SQL Server-Integration
     \item Entwickelte das Shaahin-Projekt mit ReactJS-Frontend und Hasura/PostgreSQL-Backend
     \item Entwickelte CI/CD-Pipelines und richtete Kubernetes-Bereitstellungsserver ein
     \item Erstellte Kubernetes-Manifeste, Helm-Charts und verwaltete Infrastruktur mit Terraform
 \end{subsection}
\end{section}

\newpage

\begin{section}{Bildung}
 \begin{subsectionnobullet}{M.Sc in Informatik}{Universität Teheran}{Sep 2021 -- Aug 2024}{Teheran, Iran}
     \item Thesis: Erkennung von Anfällen aus Mehrkanal-EEG mit Graph Neural Networks.
 \end{subsectionnobullet}
 \begin{subsectionnobullet}{B.Sc in Informatik}{Kharazmi Universität}{Sep 2016 -- Jan 2021}{Teheran, Iran}
     \item Abschluss mit starker akademischer Leistung.
 \end{subsectionnobullet}
\end{section}

\sectiontable{Technische Fähigkeiten}{
    \entry{Programmiersprachen}{C/C++, C\#, Java, JavaScript, TypeScript, Python, MATLAB, Erlang}
    \entry{Speicher \& Datenbanken}{PostgreSQL, MSSQL, MySQL, ScyllaDB, SQLite, RabbitMQ, Redis, Minio}
    \entry{Container-Orchestrierung}{Docker, DockerSwarm, Kubernetes (K3s, RKE2, EKS), Helm, Nomad}
    \entry{Hochleistungsrechnen}{Slurm, SunGridEngine, Rocks Cluster, NvidiaEnroot, NvidiaPyxis, Singularity}
    \entry{Infrastruktur \& Cloud}{Terraform, Ansible, Packer, ESXi, Hetzner Cloud, Amazon Web Services}
    \entry{Überwachung \& Alarmierung}{Grafana, Promtail, Prometheus, OpenTelemetry, Loki, Mimir, Tempo}
    \entry{CI/CD \& Automatisierung}{DroneCI, GitlabCI, Github Actions, Azure Pipelines, ArgoCD, GitOps}
    \entry{LowCode \& NoCode}{N8n, Hasura, Strapi, ReactAdmin}
}

\sectiontable{Sprachen}{
    \entry{Persisch}{Muttersprache}
    \entry{Englisch}{Berufliche Verhandlungssicherheit}
    \entry{Deutsch}{Grundkenntnisse}
}

\end{document}
