\documentclass[]{main}
\fullname{Mohammad Hossein Nemati}
\jobtitle{DevOps Ingenieur}

\begin{document}
\resumeheader
{\email{mhmnemati@gmail.com}}
{\website{mhmnemati.com}}
{\linkedin{mhmnemati}}
{\github{mhmnemati}}
{}
{}

Ich bin ein DevOps-Ingenieur mit über 5 Jahren Erfahrung, spezialisiert auf die Implementierung von CI/CD-Pipelines, Dockerisierung und Infrastructure as Code (IaC) in Teams von mehr als 10 Ingenieuren. Ich habe die Bereitstellung für Systeme optimiert, die täglich Millionen von Transaktionen verarbeiten. Während mein Fokus auf DevOps liegt, beschäftige ich mich auch mit MLOps, insbesondere mit der Automatisierung von Modelltraining und -inferenz.

\begin{section}{Berufserfahrung}
 \begin{subsection}{Visiwise.co}{Senior DevOps Ingenieur}{Mai 2022 -- Gegenwart}{}
     \item Refaktorisierte Codebasen in über 10 Projekten und richtete CI-Pipelines ein, wodurch sich die Bereitstellungszeiten um 70\% reduzierten
     \item Migrierte über 10 Dienste in einen Docker Swarm Cluster auf Hetzner Cloud, was die Ressourceneffizienz um 25\% verbesserte
     \item Clusterte PostgreSQL und automatisierte Backups, um 100\% Datenintegrität während der Migration ohne Ausfallzeit sicherzustellen
     \item Automatisierte die Bereitstellung von Infrastrukturen mit Terraform und Ansible in über 5 Umgebungen
     \item Bereitstellte über 5 selbst gehostete Dienste, wodurch die Abhängigkeit von Drittanbietern um 40\% reduziert wurde
     \item Etablierte Monitoring mit dem Grafana Stack, was die Reaktionszeit bei Vorfällen um 80\% verbesserte mit über 10 benutzerdefinierten Dashboards
     \item Standardisierte CI/CD-Pipelines, wodurch die Build-Zeiten in über 10 Repositories auf unter 1 Minute reduziert wurden
     \item Verwalte ein Multi-Cloud-Setup (Hetzner \& AWS), was die Betriebskosten um 20\% senkte
     \item Implementierte Review-Apps, wodurch die Produktivität der Entwickler um 40\% gesteigert wurde
 \end{subsection}

 \begin{subsection}{Smartech}{Senior DevOps Ingenieur}{Feb 2024 -- Juni 2024}{}
     \item Standardisierte CI/CD-Vorlagen über 50+ Repositories, was die Bereitstellungszeiten um 40\% reduzierte
     \item Konfigurierte auto-skalierende Kubernetes-Cluster mit Terraform für 10+ Event Processor Apps, was die Kafka-Lastverarbeitung optimierte
     \item Entwickelte Kubernetes Manifeste für 20+ Projekte und automatisierte die Bereitstellung von 40+ Diensten über ArgoCD
     \item Automatisierte das Servermanagement mit Ansible für 50+ Server, einschließlich Firewall, Benutzerverwaltung und Minio-Cluster
     \item Verwalte 500+ VMs und 50TB+ warme Daten über Dienste wie Elasticsearch, ClickHouse, ScyllaDB und Kafka
 \end{subsection}

 \begin{subsection}{Universität Teheran}{DevOps Ingenieur}{Juli 2022 -- Juni 2023}{}
     \item Bereitstellte einen HPC-Cluster mit über 20 Knoten mithilfe von Terraform und Ansible; richtete Slurm mit Singularity und Nvidia Enroot ein
     \item Optimierte GPU-Workloads für 10+ Forscher durch Integration von Nvidia Enroot mit Slurm über Nvidia Pyxis
     \item Verwaltete Berechtigungen und Quoten für 200+ Benutzer und erstellte 5+ ressourcenspezifische Warteschlangen für Jobs
     \item Implementierte einen Grafana Monitoring Stack, was die Ressourcenauslastung um 30\% verbesserte
 \end{subsection}

 \begin{subsection}{Pishgam Vira}{DevOps Ingenieur}{März 2022 -- Juni 2022}{}
     \item Bereitstellte Docker Swarm Cluster und Datenbankserver mit Ansible über 3+ Umgebungen hinweg
     \item Schrieb Docker Compose Manifeste für optimierte Bereitstellungen
     \item Richtete CI/CD-Pipelines für 5+ Repositories mit GitHub Actions ein
     \item Automatisierte die Versionsverwaltung mit semantic-release
 \end{subsection}

 \begin{subsection}{Mobtaker Darya}{Full-Stack Entwickler \& DevOps Ingenieur}{Nov 2019 -- Juni 2022}{}
     \item Wartete und entwickelte das CCS-Projekt weiter, was zu über 50 Berichten mit Crystal Reports und SQL Server führte
     \item Entwickelte das Shaahin-Projekt mit einem ReactJS-Frontend und Hasura/PostgreSQL-Backend
     \item Implementierte CI/CD-Pipelines mit GitLab CI und Docker, verwaltete die Infrastruktur mit Terraform und Ansible
     \item Bereitstellte Server mit Kubernetes, schrieb Manifeste und packte sie in Helm Charts
 \end{subsection}

 \begin{subsection}{Tolou Yekta Samaneh}{Full-Stack Entwickler}{März 2019 -- Okt 2019}{}
     \item Wartete und refaktorisierte das Managed-Funds-System (MFD), implementierte Backend-APIs mit Loopback.io, Node.js, TypeScript und PostgreSQL
     \item Ersetzte Process360 durch ProcessMaker als BPMS-Kern und entwickelte eine neue Benutzeroberfläche mit ReactJS und TypeScript
     \item Implementierte Formulargeneratoren im Client mit Schema-Integration aus BPMS
     \item Entwickelte eine Lebensmittel-Einkaufs-Benutzeroberfläche für das Farmeal-Projekt mit ReactJS, MaterialUI und TypeScript und integrierte sie mit Backend-GraphQL-APIs mithilfe von Apollo Client
     \item Richtete CI/CD-Pipelines mit GitLab CI und Docker für die ManagedFunds- und Farmeal-Projekte ein
 \end{subsection}

 \begin{subsection}{ChaM (Chapar Messenger)}{Gründer \& Lead Entwickler}{Dez 2013 -- Aug 2018}{}
     \item Entwarf und entwickelte eine Audio/Video-Messaging-Plattform mit C, Erlang und Java
     \item Implementierte P2P-Verbindungen und nutzte Cassandra/ScyllaDB für das Chat-Datenmanagement
     \item Erstellte einen Desktop-Client mit Qt/QML (C++) und eine Android-App mit Java und JNI
     \item Integrierte FFMPEG für Audio/Video-Streaming mit nativen APIs
     \item Verwaltete die Full-Stack-Entwicklung, einschließlich Serverinfrastruktur und Client-Schnittstellen
 \end{subsection}
\end{section}

\sectiontable{Technische Fähigkeiten}{
    \entry{Programmiersprachen}{C/C++, C\#, Java, JavaScript, TypeScript, Python, MATLAB, Erlang}
    \entry{Speicher \& Datenbanken}{PostgreSQL, MSSQL, MySQL, ScyllaDB, SQLite, RabbitMQ, Redis, Minio}
    \entry{Container-Orchestrierung}{Docker, DockerSwarm, Kubernetes (K3s, RKE2, EKS), Helm, Nomad}
    \entry{Hochleistungsrechnen}{Slurm, SunGridEngine, Rocks Cluster, NvidiaEnroot, NvidiaPyxis, Singularity}
    \entry{Infrastruktur \& Cloud}{Terraform, Ansible, Packer, ESXi, Hetzner Cloud, Amazon Web Services}
    \entry{Monitoring \& Alarmierung}{Grafana, Promtail, Prometheus, OpenTelemetry, Loki, Mimir, Tempo}
    \entry{CI/CD \& Automatisierung}{DroneCI, GitlabCI, Github Actions, Azure Pipelines, ArgoCD, GitOps}
}

\newpage

\begin{section}{Bildung}
 \begin{subsectionnobullet}{M.Sc in Informatik}{Universität Teheran}{Sep 2021 -- Aug 2024}{}
     \item Thesis: Anfallserkennung aus mehrkanaligem EEG mit Graph Neural Networks
 \end{subsectionnobullet}
 \begin{subsectionnobullet}{B.Sc in Informatik}{Kharazmi Universität}{Sep 2016 -- Jan 2021}{}
     \item Mit herausragender akademischer Leistung abgeschlossen
 \end{subsectionnobullet}
\end{section}

\sectiontable{Sprachen}{
    \entry{Persisch}{Muttersprache}
    \entry{Englisch}{Verhandlungssicher}
    \entry{Deutsch}{Grundkenntnisse}
}

\end{document}
