\documentclass[]{main}
\fullname{Mohammad Hossein Nemati}
\jobtitle{Senior DevOps Engineer}

\begin{document}
\resumeheader
{\address{Leipzig, Deutschland}}
{\phone{+49 152 01847345}}
{\email{ckoliber@gmail.com}}
{\website{ckoliber.dev}}
{\linkedin{ckoliber}}
{\github{ckoliber}}

Senior DevOps Engineer mit mehr als 5 Jahren Erfahrung im Aufbau skalierbarer, automatisierter Infrastrukturen in Multi-Cloud-Umgebungen. Ich standardisiere CI/CD (über 50 Repositories), stärke Hochleistungs-Deployments und betreibe zentralisierte Observability (Grafana Stack). Ich erforsche MLOps zur Automatisierung von Modell-Training und -Inferenz.

\begin{section}{Berufserfahrung}
 \begin{subsection}{Visiwise.co}{Senior DevOps Engineer}{Mai 2022 -- Heute}{}
     \item Reduzierte Infrastrukturkosten durch Migration von Workloads zu ARM64 auf Hetzner Cloud
     \item Entwarf skalierbare Swarm-Cluster für Staging und Produktion unter Verwendung von Terragrunt, Terraform und Ansible
     \item Verbesserte Deployment-Stabilität in über 10 Projekten durch Refactoring nach den 12-Factor-Praktiken
     \item Etablierte Postgres HA mit PgCat; automatisierte Backups und Zero-Downtime-Migrationen
     \item Zentralisierte Observability mit Grafana/Loki/Prometheus; richtete Grafana OnCall mit Eskalationsrichtlinien ein
     \item Standardisierte CI/CD; fügte Review-Apps, geplante E2E-Tests hinzu und aktivierte Renovate für private Registries
     \item Migrierte die Uptime-Überwachung zu BetterStack mit Eskalationsketten (Anruf/SMS)
     \item Automatisierte die Provisionierung über verschiedene Umgebungen hinweg; hostete Sentry und GitLab selbst, um die SaaS-Ausgaben zu senken
     \item Migrierte Kerndienste zu Google Cloud Platform (GCP); betrieb Multi-Cloud über AWS, GCP und Hetzner
 \end{subsection}

 \begin{subsection}{Faraatar Entrepreneurship Group}{Senior DevOps Engineer (Vertrag)}{Juli 2024 -- Okt 2024}{}
     \item Erstellte eine mandantenfähige CRM-SaaS-Plattform mit Kubernetes und Terraform
     \item Standardisierte CI/CD-Pipelines für GitLab-Projekte und -Umgebungen
     \item Lieferte eine lizenzierte CRM-Laufzeitumgebung für On-Premise-Kunden mit Apptainer und verschlüsselten Containern
     \item Entwickelte einen Golang-Agenten, um die neuesten lizenzierten Images sicher abzurufen und auszuführen, während der Quellcode geschützt wurde
 \end{subsection}

 \begin{subsection}{Smartech}{Senior DevOps Engineer}{Jan 2024 -- Sep 2024}{}
     \item Standardisierte CI/CD-Templates in über 50 Repositories zur Vereinheitlichung der Deployment-Prozesse
     \item Bereitstellung und Verwaltung von Kafka-Clustern mit Ansible zur Sicherstellung einer zuverlässigen Ereignisverarbeitung
     \item Betrieb von ClickHouse-Clustern mit über 50 TB Zeitreihendaten für Echtzeitanalysen
     \item Automatisierte ClickHouse-Backups und handhabte Schema-Migrationen in der Produktion
     \item Erstellte einen automatisch skalierbaren RKE2-Cluster auf Hetzner mit Terraform und Ansible
     \item Integrierte Cluster Autoscaler zur dynamischen Skalierung von Nodes basierend auf der Arbeitslast
     \item Entwickelte einen Custom Pod Autoscaler (CPA) zur Skalierung von Workern basierend auf der Kafka Consumer Lag
     \item Erstellte Kubernetes-Manifeste für über 20 Mikroservices und verwaltete Deployments über ArgoCD
     \item Automatisierte die Serververwaltung für über 50 Nodes mit Ansible, einschließlich Firewall- und Benutzer-Setups
     \item Wartete über 100 VMs und stellte die Verfügbarkeit von Kafka-, ClickHouse- und ScyllaDB-Clustern sicher
 \end{subsection}

 \begin{subsection}{Universität Teheran}{DevOps Engineer}{Juli 2022 -- Juni 2023}{}
     \item Provisionierte einen 20+ Node HPC-Cluster mit Terraform, Ansible, Slurm, Singularity und Nvidia Enroot
     \item Optimierte GPU-Workloads für über 10 Forscher durch Integration von Nvidia Enroot mit Slurm über Nvidia Pyxis
     \item Verwaltete Berechtigungen und Quoten für über 200 Benutzer und erstellte über 5 ressourcenspezifische Warteschlangen für Jobs
     \item Implementierte einen Grafana-Monitoring-Stack zur Überwachung von Cluster-Ressourcen und Workloads
 \end{subsection}

 \begin{subsection}{Pishgam Vira}{DevOps Engineer}{März 2022 -- Juni 2022}{}
     \item Provisionierte Docker Swarm-Cluster und Datenbankserver mit Ansible in über 3 Umgebungen
     \item Erstellte Docker Compose-Manifeste für optimierte Deployments
     \item Richtete CI/CD-Pipelines für über 5 Repositories mit GitHub Actions ein
     \item Automatisierte die Versionierung mit semantic-release
 \end{subsection}

 \begin{subsection}{Mobtaker Darya}{Full-stack Entwickler \& DevOps Engineer}{Nov 2019 -- Juni 2022}{}
     \item Wartete und entwickelte das CCS-Projekt, das über 50 Berichte mit Crystal Reports und SQL Server generierte
     \item Entwickelte das Shaahin-Projekt mit einem ReactJS-Frontend und einem Hasura/PostgreSQL-Backend
     \item Implementierte CI/CD-Pipelines mit GitLab CI und Docker, IaC mit Terraform und Ansible
     \item Provisionierte Server mit Kubernetes, schrieb Manifeste und verpackte sie in Helm Charts
 \end{subsection}

 \begin{subsection}{ChaM (Chapar Messenger)}{Gründer \& Leitender Entwickler}{Dez 2013 -- Aug 2018}{}
     \item Entwarf und entwickelte eine Audio-/Video-Messaging-Plattform mit C, Erlang und Java
     \item Implementierte P2P-Verbindungen und nutzte Cassandra/ScyllaDB für die Chat-Datenverwaltung
     \item Erstellte einen Desktop-Client mit Qt/QML (C++) und eine Android-App mit Java und JNI
     \item Integrierte FFMPEG für Audio-/Video-Streaming mit nativen APIs
     \item Verwaltete die Full-Stack-Entwicklung, einschließlich Server-Infrastruktur und Client-Interfaces
 \end{subsection}
\end{section}

\sectiontable{Technische Fähigkeiten}{
    \entry{Programmiersprachen}{C/C++, C\#, Java, JavaScript, TypeScript, Python, Erlang, Golang, MATLAB}
    \entry{Speicher \& Datenbanken}{PostgreSQL, MSSQL, MySQL, SQLite, ScyllaDB, ClickHouse, RabbitMQ, Kafka, Redis, MinIO}
    \entry{Container-Orchestrierung}{Podman, Docker, Docker Compose, Docker Swarm, Kubernetes (K3s, RKE2, EKS), Helm, Nomad}
    \entry{High Performance Computing}{Slurm, Sun Grid Engine, Rocks Cluster, Nvidia Enroot, Nvidia Pyxis, Singularity, Apptainer}
    \entry{Infrastruktur \& Cloud}{Terragrunt, Terraform, Ansible, Packer, ESXi, OpenStack, Hetzner Cloud, GCP, AWS}
    \entry{Überwachung \& Alarmierung}{Grafana, GrafanaOnCall, Prometheus, Promtail, OpenTelemetry, Loki, Mimir, Tempo, Alloy}
    \entry{CI/CD \& Automatisierung}{DroneCI, GitlabCI, TravisCI, Github Actions, Azure Pipelines, ArgoCD, GitOps}
}

\begin{section}{Ausbildung}
 \begin{subsectionnobullet}{M.Sc in Informatik}{Universität Teheran}{Sep 2021 -- Aug 2024}{}
     \item Abschlussarbeit: Epilepsie-Erkennung aus Mehrkanal-EEG mithilfe von Graph Neural Networks
 \end{subsectionnobullet}
 \begin{subsectionnobullet}{B.Sc in Informatik}{Kharazmi Universität}{Sep 2016 -- Jan 2021}{}
     \item Abschluss mit sehr guter akademischer Leistung
 \end{subsectionnobullet}
\end{section}

\sectiontable{Sprachen}{
    \entry{Persisch}{Muttersprache}
    \entry{Englisch}{Verhandlungssicher}
    \entry{Deutsch}{Grundkenntnisse}
}

\end{document}
